\chapter{Выбор инструментов} 

\section{Язык программирования}
Выбор языка программирования --- важное решение, влияющее на сложность написания и поддержки системы, возможности масштабирования, а также ограничения эксплуатации программы.

Далее выделим требования к языку общего назначения, на котором будет осуществляться разработка программного комплекса.
\begin{enumerate}
%	\item Поддержка функциональной парадигмы программирования;
	\item Возможность написания кроссплатформенных программ. Кроссплатформенность --- способность ПО работать более чем на одной аппаратной платформе и операционной системе.
	\item Безопасность ~--- набор характеристик, определяющих то, насколько рано технические средства работы с программой на конкретном языке могут определить ошибку. 
	\begin{enumerate}
		\item Статическая типизация — приём, широко используемый в языках программирования, при котором переменная, параметр подпрограммы, возвращаемое значение функции связывается с типом в момент объявления и тип не может быть изменён позже (переменная или параметр будут принимать, а функция — возвращать значения только этого типа).
		\item Валидация во время компиляции.
	\end{enumerate}
	\item Высокий уровень языка.
	\item Поддержка рефлексии. Рефлексия (англ. reflection)~--- процесс, во время которого программа может отслеживать и модифицировать собственную структуру и поведение во время выполнения.
	\item Совместимость с другими языками программирования. 
	\item Краткость. Выражается возможности автоматической генерации кода, использования $ \lambda $-нотации для описания функции, вывод типов данных.
	\item Возможность описания предметно-ориентированных языков (не обязательное, но желательное требование).
\end{enumerate}

\subsection{С++}
C++~--- компилируемый, статически типизированный язык программирования общего назначения. 
Язык поддерживает такие парадигмы программирования, как процедурное программирование, объектно-ориентированное программирование, обобщённое программирование. 
Язык имеет богатую стандартную библиотеку, которая включает в себя распространённые контейнеры и алгоритмы, ввод-вывод, регулярные выражения, поддержку многопоточности и другие возможности. 
В сравнении с его предшественником, языком C, наибольшее внимание уделено поддержке объектно-ориентированного и обобщённого программирования.

Для обеспечения кроссплатформенности программы необходимо использовать кроссплатформенные фреймворки, которых, к слову, становится всё больше и больше.
Программист C++ вынужден контролировать выделение и освобождение памяти, что является отрицательным для фактора безопасности, т.к. вносит значительную вероятность ошибки, источник которой отследить крайне сложно.

Язык не предоставляет поддержку рефлексии, однако существует расширения языка Qt, позволяющего осуществлять доступ к специально описанным объектам системы, наследникам QObject.

C++ сочетает в себе свойства языков высокого и низкого уровней.

Код на C++ достаточно громоздкий, а возможности кодогенерации ограничены использованием макросов~--- шаблонов кода, определённых для препроцессора. 
Однако, язык позволяет использовать $ \lambda $-выражения для описания функций.

\subsection{Java}

Java~--- сильно типизированный объектно-ориентированный язык программирования, разработанный компанией Sun Microsystems (в последующем приобретённой компанией Oracle). 
Приложения Java обычно транслируются в специальный байт-код, поэтому они могут работать на любой компьютерной архитектуре с помощью виртуальной Java-машины.
Также возможна ahead-of-time компиляция в машинный код частей программ или программ целиком.

Язык достаточно безопасен, компилятор осуществляет валидацию кода и способен детектировать большую часть ошибок, однако в нём отсутствуют некоторые выразительные возможности, связанные с типами. 
Основной проблемой считается null-уязвимость типа. 
Проблема заключается в отсутствии возможности указать компилятору, может ли переменная принимать значение null и при запрете, выдавать ошибку при некорректном присвоении значения.

Ещё одним плюсом для безопасности использования является модель памяти Java, в которой отсутствует необходимость явно выделять/освобождать память объектов. 
Java содержит библиотеку поддержки рефлексии, совместима со статическими библиотеками, написанными на C, а также поддерживает $ \lambda $-нотацию.

Программа на языке Java не будет сильно отличаться по размеру от аналогичной, написанной на C++. 
Однако язык позволяет писать расширения компилятора, генерирующие байт-код. 
Существует много библиотек, позволяющих писать исходный код более кратко, а также дополняющих язык различными проверками при компиляции.

Java не позволяет описать DSL.

\subsection{Kotlin}

Kotlin (Котлин)~--- статически типизированный язык программирования высокого уровня, работающий на JVM (Java Virtual Machine) и разрабатываемый компанией JetBrains. 
Кроме компиляции в байткод, возможна компиляция в JavaScript и на другие платформы через инфраструктуру LLVM. 
Язык назван в честь острова Котлин в Финском заливе, на котором расположен город Кронштадт.
Авторы ставили целью создать язык более лаконичный и типобезопасный, чем Java. 

Язык Kotlin один из самых безопасных на сегодняшний день\cite{vakhitov2016}.
Помимо статической типизации он также решает проблему null-уязвимости типа данных.
На уровне языка поддерживается много шаблонов проектирования, в т.ч. шаблоны <<одиночка>>, <<builder>>, DSL.
Также язык Kotlin очень ёмкий и предоставляет много удобных операторов.
Поддерживает $ \lambda $-выражения.

Kotlin совместим с Java.
Возможно построение мультиплатформенных приложений, когда одна и та же модель данных используется для серверной и клиентской частей приложения на базе C++, Java или JavaScript.

\subsection{Итог}

Учитывая обозначенные выше критерии, можно сделать вывод, что язык Kotlin подходит для реализации управляющей программы лучше всего.
\begin{table} [htbp]
	\centering
	\parbox{15cm}{%
		\caption{Сравнение языков общего назначения}\label{Ts0Sib}%
	}
\begin{tabular}{|c|c|c|c|}
			\hline
			Язык & С++ & Java & Kotlin \\ \hline
			Кроссплатформенный & +- & + & + \\ \hline
			Статическая типизация & + & + & + \\ \hline
			\begin{tabular}{@{}c@{}}Валидация при \\ компиляции\end{tabular}
			 & +- & +- & + \\ \hline
			Уровень & высокий + низкий & высокий &  высокий \\ \hline
			Рефлексия & - & + & + \\ \hline
			 \begin{tabular}{@{}c@{}}Совместимость с\\ другими языками\end{tabular}  & Assembler & C++, Kotlin & 			\begin{tabular}{@{}c@{}}Java, C++,\\ JavaScript \end{tabular}\\ \hline
			Краткость & 
			\begin{tabular}{@{}c@{}}Исходный код \\ очень объёмен\end{tabular} & \begin{tabular}{@{}c@{}}Исходный код \\ очень объёмен\end{tabular} & Код краток \\ \hline
			DSL & - & - & +\\ \hline
			
\end{tabular}
\end{table}
\section{Прочие инфраструктурные решения}

Существует много библиотек на языке Kotlin, облегчающих взаимодействие компонентов программы и предлагающих свои DSL для различных областей, включая проектирование графических интерфейсов.

В качестве инфраструктурного решения широко применяется техника внедрения зависимостей.
Внедрение зависимости (англ. Dependency injection, DI)~--- процесс предоставления внешней зависимости программному компоненту. 
Является специфичной формой <<инверсии управления>> (англ. Inversion of control, IoC), когда она применяется к управлению зависимостями. 
В полном соответствии с принципом единственной ответственности~\cite{martin2002agile} объект не участвует в построении требуемых ему зависимостей. 
Существует специально предназначенный для построения и передачи зависимостей общий механизм.
Самой надёжной и наиболее функциональной библиотекой для его осуществления на языке Kotlin является библиотека Spring\cite{reddy2017spring}.
Помимо основной цели, эта библиотека имеет ряд опциональных модулей, позволяющих быстро интегрировать различные технологии в приложение. 
Например, библиотека spring-boot-web позволяет добавить в приложение web интерфейс и реализовать клиент-серверную архитектуру.

В первом приближении планируется создать оконное приложение, однако обилие доступных интерфейсов взаимодействия, а также лёгкость создания новых интерфейсов, позволит интегрировать систему со сторонними приложениями.

Для создания оконного приложения можно использовать ряд специализированных библиотек, например, встроенные библиотеки Java: Swing(устаревшая) или JavaFX.
Оба решения достаточно громоздкие ввиду своей гибкости.
Однако и эта проблема решена в Kotlin. 
Библиотека TornadoFX \cite{dea2017appendix}  позволяет ёмко описывать интерфейс на предоставляемом DSL и транслирует его в код, использующий JavaFX во время компиляции.