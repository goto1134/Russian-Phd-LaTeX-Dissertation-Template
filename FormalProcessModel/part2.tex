\chapter{Требования к реализации}
\section{Постановка задачи разработки программного комплекса}

Необходимо построить программный комплекс, реализующий описанный формализм параллельных вычислительных процессов.

Решение должно состоять из следующих компонентов:
\begin{itemize}
	\item язык описания системы;
	\item управляющая программа;
	\item модуль визуализации.
\end{itemize}


\section{Язык описания системы}
Задача построения динамических систем всё чаще возникает в современной разработке. 
В общем случае фундаментом такой разработки является язык общего назначения, применимый к широкому спектру областей и не учитывающий особенности конкретной сферы знаний\cite{berry1989real}.
Этот инструмент, вместе с гибкостью, приносит и массу проблем, связанных с резко возрастающей сложностью реализации и сопровождения системы в процессе роста программного комплекса.
Альтернативой могут служить предметно-ориентированные языки.

Предметно-ориентированный язык (англ. domain-specific language, DSL)~--- язык программирования, специализированный для конкретной области применения.
Такие языки часто встречаются в вычислениях.
\begin{itemize}
	\item CSS (англ. Cascading Style Sheets)~--- формальный язык описания внешнего вида документа, написанного с использованием языка разметки.
	\item Регулярные выражения (англ. regular expressions)~--- формальный язык поиска и осуществления манипуляций с подстроками в тексте, основанный на использовании метасимволов (символов-джокеров, англ. wildcard characters).
	\item Различные средства автоматизации сборки и тестирования программ: make, rake, ant, gradle.
	\item SQL (англ. Structured Query Language)~--- язык программирования, применяемый для создания, модификации и управления данными в реляционной базе данных, управляемой соответствующей системой управления базами данных.
	\item и т.д.
\end{itemize}
Все они позволяют разработчику описывать систему, используя термины предметной области \cite{van2000domain}.

В нашем случае такой язык должен включать сведения о структуре системы и её подсистем, о возможных типах данных, а также об исполняемых процедурах. 
Последнее возможно реализовать, например, за счёт ссылки на файл подпрограммы, или инъекцией сторонних языков.

DSL могут быть как текстовыми, так и графическими. 
Для графических языков требуется дополнительное программное обеспечение, позволяющее редактировать графическое представление.

Предполагается создание как текстового языка, так и графического расширения \cite{sprinkle2004domain} для него, позволяющего манипулировать структурными данными.

Для исполнения программы DSL должен быть преобразован в представление задачи на языке управляющей программы.
\section{Управляющая программа}

Управляющая программа~--- реализация на языке программирования общего назначения процесса исполнения описанной вычислительной системы.
Управляемая пользователем, она должна производить вычисления и обеспечивать доступ к информации о процессе вычисления и состоянии системы.
Правила и алгоритмы исполнения описаны в первой главе.
В конце работы такой программы должен быть осуществлён вывод результата вычислений, а также, при необходимости, запрошенные метрики процесса: время, максимальный использованный объём памяти и другие.


\section{Модуль визуализации}
Проектирование систем параллельных вычислений~--- сложная и трудоёмкая задача, требующая от исполнителя глубокого понимания структуры и процессов, возможных и происходящих в системе. 
Облегчить её может визуализация структуры и состояния вычислительной сети.

Модуль визуализации~--- клиентская программа, позволяющая следить за ходом исполнения вычислительной задачи, предоставляющая необходимые метрики, а также возможность влияния на процесс вычислений.

Таким образом модуль визуализации должен отображать:
\begin{itemize}
	\item текущий интерфейс вычислительной сети,
	\item кнопки для управления процессом вычисления,
	\item результаты вычислений,
	\item значения, графики и сводные таблицы метрик выполнения вычислений для оценки реализации.
\end{itemize}

В рамках данной работы предполагается реализовать управляющую программу и язык описания вычислительной системы для статической системы (без нетерминальных элементов) с последовательным интерфейсом.
В качестве функционального базиса выбран наиболее гибкий вариант, в котором результатом функции является кортеж эпизодов неопределённого размера.
Визуализацию процесса же достаточно реализовать в виде последовательности изображений - <<снимков>> системы без интерактивной составляющей и метрик. 
Однако в проекте необходимо учесть дальнейшее расширение функций.