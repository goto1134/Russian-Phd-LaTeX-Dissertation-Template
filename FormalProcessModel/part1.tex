\chapter{Теоретическое введение}
\section{Процессы реального времени}
Абстрактный процесс в системе определяется как последовательность $ \langle P_0,P_1,...\rangle $   состояний системы, а процесс реального времени – как последовательность упорядоченных пар $ \langle P_i,t_i \rangle, i\in N_0 $\footnote{$ N_0 \cong \{0,1,2,\dots\}$} представленных состояниями системы и временами переходов системы в эти состояния, причем времена в последовательности строго упорядочены по возрастанию. 
Далее мы будем рассматривать только процессы реального времени. 
В общем случае для недетерминированных систем процессы представлены конечными или бесконечными путями из корня дерева процессов.
Корень дерева процессов представляет собой пару $ \langle P(t_0),0 \rangle $ , где $ P_0 $  – начальное состояние всех заданных этим деревом процессов в момент времени  $ t_0 = 0 $. 
В понятии процесса используется понятие времени, как абстрактного, представленного номерами в образующей процесс последовательности, так и реального, представленного элементами числовой шкалы времени, которая представлена линейно упорядоченным множеством моментов времени и, желательно: 
\begin{itemize}
	\item включает минимальное и максимальное значения, 
	\item имеет разрешимые предикаты сравнения ($ <,\leq,=,\geq, > $),
	\item является плотной, в которой для любых различных моментов времени $ t^{'} $  и $ t^{''} $  , таких, что $ t^{'}<t^{''} $ , существует момент времени $ t $ , такой, что  $ t^{'}<t $  и $ t<t^{''} $ .
\end{itemize}

Не ограничивая общности, будем в качестве удовлетворяющей этим требованиям шкалы времени рассматривать множество $ \breve{Q} = Q \cup \{\omega\} $, где $ Q=Q_+\cup{0} $ ,$ Q_+ $~--- множество положительных рациональных чисел, $ 0 $ --- минимальное значение:$ (\forall t \in Q_+) (0<t) $ , $ \omega $--- несобственное значение, время, неограниченно отдаленное в будущем: $ (\forall t \in Q) (t<\omega) $. 
Сохраняя основные свойства арифметических операций, доопределим естественным образом для элементов множества $ \breve{Q} $ указанные отношения $ <,\leq,=,\geq, > $ и арифметические операции сложения, вычитания, умножения и деления, за исключением некоторых случаев: $ \omega/\omega $ , $ 0/0 $ , $ 0*\omega $  и вычитания из меньшего значения большего. 
Заметим, что положительные рациональные числа представляются упорядоченными парами положительных натуральных чисел, в общем случае неоднозначно. Каноническое представление предполагает, что эти числа являются взаимно-простыми. Через $ \tau $   будем далее обозначать текущее время в системе, через $ t\in Q $  --- произвольный момент времени.  

\section{Понятие эпизода}
В основе предлагаемых далее определений мы используем понятие эпизода. 
Эпизоды задаются их типами и двумя моментами времени: начала и завершения (более точно определено ниже). 
Неформально, эпизоды~--- некие сущности (объекты, свойства и т.п.) на интервалах рассматриваемой шкалы времени. 
Интервал~--- упорядоченная пара $ \langle t^{'},t^{''}\rangle $ моментов времени, такая, что  $ t^{'}<t^{''} $. 
Состояние системы в любой момент времени $ t $ определяется как неупорядоченный набор (комплект, конечное мультимножество) не завершившихся к этому времени эпизодов.

Пусть $ p\equiv \langle e,\theta \rangle $~--- эпизод. 
Первый компонент представляет тип эпизода из не более чем счетного множества  $ E $  типов эпизодов в рассматриваемой системе. 
Важную роль в описании частных случаев определения процессов в системах играет структуризация множества типов эпизодов. 
Разбиение типа на два компонента практически не ограничивает общности: $ E \equiv E_B \times E_D $. 
Первый компонент типа определяет возможные события в системе, приводящие к изменению ее состояния, а второй~--- наполняет их конкретным содержанием. 
Для всех  $ \langle e_B,e_D \rangle \in E $  полагаем, что $ e_B $~--- структурный атрибут эпизода, $ e_D $~--- его информационный атрибут. 
Практически полезной является и дальнейшая детализация структурного атрибута: полагаем, что $ e_B \equiv \langle e_B^{'}, e_B^{''} \rangle $ , где $ e_B^{'} $~--- один из группирующих эпизоды контейнеров, к которому <<относится>> (или, иначе говоря, в котором <<находится>> эпизод), а $ e_B^{''} $~--- сорт (или тег) эпизода, уточняющий средства оперирования с информационными атрибутами (аналог понятия типа или класса информационных объектов в языках программирования). 
В частном случае, сорт эпизода может однозначно определяться его контейнером. 
В свою очередь, сорт эпизода определяет множество возможных значений его информационного атрибута. 
Предполагается, что множество значений информационного атрибута любого сорта содержит неопределенное значение  $ \perp $.

Второй компонент эпизода $ \theta \equiv \langle t^{'} t^{''} \rangle \in Q \times \breve{Q} $, $ t^{'}<t^{''} $ задает временные характеристики конкретного эпизода: $ t^{'}\in Q $  определяет начало эпизода на шкале времени; $ t^{''}\in \breve{Q} $ , если эпизод завершен ($ \tau \geq t^{''} $), то это --- конец эпизода, иначе эпизод полагается незавершенным, и $ \delta \cong t^{''}-t^{'} > 0 $ понимается как предельно возможная длительность незавершенного эпизода (если $ t^{''}=\omega $ , то возможная длительность эпизода сверху не ограничена). 

\section{События}

Изменение состояний системы в автоматной модели определяется как ее текущим состоянием, так и действующим интерфейсом, который устанавливает взаимосвязь эпизодов и, в общем случае, сам тоже может изменяться при переходах системы к новым состояниям. В каждый момент времени действующий интерфейс определяет для текущего состояния системы возможность событий, в результате которых могут измениться и состояние системы, и действующий интерфейс. 
Событие~--- переход системы в новое состояние, который характеризуется моментом времени этого перехода и определяется временем ближайшего предшествующего события, текущим состоянием системы и действующим интерфейсом эпизодов. 
Первопричинами основных событий являются: 
\begin{enumerate}[label=1)]
	\item появление в состоянии системы наборов (комплектов) эпизодов, необходимые требования к которым и определяет действующий интерфейс;
	\item в рассматриваемых далее динамических системах~--- подключение к системе определенным образом модифицированных копий подсистем. 
\end{enumerate}
Изменение состояния системы возможно также в связи с достижением максимально возможного времени окончания некоторого эпизода, в этом случае такой эпизод исключается из состояния системы и не может влиять на последующие события. 
Информационный атрибут такого завершенного эпизода полагается равным $ \perp $.

\section{Интерфейс системы}
Согласно нашему первоначальному замыслу, основные объекты в определении понятия системы~--- состояние системы, интерфейс, элемент интерфейса – должны были быть определены как неупорядоченные наборы, соответственно, незавершенных эпизодов, элементов интерфейса и требований к структурным атрибутам эпизодов, необходимых для срабатывания элементов интерфейса, составляющих события в системе, приводящие к изменению ее состояния. 
Однако, вследствие этого появляется значительная неопределенность в поведении системы, выражающаяся в чрезмерном разрастании дерева процессов. Для предотвращения этого мы отказались от концепции неупорядоченности, что позволило ввести в описание систем разного рода приоритеты, устраняющие во многом недетерминированность поведения систем. 
Фактически, неоднозначность появляется только вследствие маловероятного случайного совпадения времени готовности к срабатывания структурно не связанных элементов интерфейса и по причине сознательного отказа от учета информационных атрибутов эпизодов в формировании событий в системе, оставив только их влияние на результат~--- возможные изменения состояния системы.

В дальнейшем будем различать два варианта систем: с последовательным и параллельным интерфейсом.

В общем случае интерфейс рассматривается как упорядоченное множество элементов интерфейса, причем требование упорядоченности используется, во-первых, для приоритетного применения элементов интерфейса для систем с последовательным интерфейсом, во-вторых, для приоритетного распределения эпизодов состояния системы при одновременном срабатывании комплекта элементов интерфейса для систем с параллельным интерфейсом.
Каждый отдельный элемент интерфейса определяет кортеж требований к элементам соответствующего ему набора эпизодов, а именно, к их контейнерам и (или) сортам, а также к временам их появления (к началам эпизодов) по отношению к минимально возможному моменту предполагаемого их совместного срабатывания – в виде так называемой <<выдержки>> эпизодов с областью значений $ Q_+ $ . 
Если из элементов текущего состояния системы можно сформировать соответствующий элементу интерфейса кортеж эпизодов, то будем говорить, что он готов к срабатыванию в рассматриваемый момент времени. 
Ограничить разнообразие удовлетворяющих этим требованиям комплектов эпизодов можно дополнительно заданием для конкретного элемента интерфейса одной из двух основных дисциплин выбора эпизодов из текущего состояния системы: FIFO – предпочтение отдается <<старым>> эпизодам (с более ранним началом) или LIFO  предпочтение отдается <<молодым>> эпизодам (с более поздним началом). 
В простейшем случае элементы действующего в текущий момент времени интерфейса могут быть заданы путем их непосредственного перечисления, разумеется, если их конечное число. 
В более общем случае, который в этой статье не рассматривается, интерфейс задается в форме некоторой схемы – выражений формального языка, содержащих, возможно, свободные вхождения переменных с известными областями значений, причем возможными значениями этих выражений являются отдельные элементы интерфейса, а само множество значений является рекурсивно-перечислимым (и, в общем случае, может быть бесконечным). 

Для систем с последовательным интерфейсом отдельные его элементы анализируются на возможность срабатывания в порядке их перечисления, вплоть до первого готового к срабатыванию в ближайший момент времени по отношению к времени предыдущего события. Очевидно, что в один и тот же момент времени может произойти последовательное срабатывание нескольких элементов интерфейса в порядке их перечисления.

Для систем с параллельным интерфейсом отдельные элементы интерфейса определяют только возможность событий в системе. В системе выделяются готовые к одновременному срабатыванию в некоторый момент времени подкомплекты эпизодов текущего состояния системы при условии ненаступления до этого других событий. Поэтому переход к новому состоянию системы, как и для систем с последовательным интерфейсом, может стать реальным только для возможных событий с минимальным ожидаемым временем. Если это время окажется больше максимально возможного времени окончания хотя бы одного из выбранных эпизодов, то для определения готовности всё следует повторить для состояния, в котором удалены такие эпизоды. Если требуемых комплектов эпизодов нет, то процесс завершается; в противном случае рассматривается готовность к одновременному срабатыванию всевозможных комплектов указанных комплектов эпизодов для отдельных элементов параллельного интерфейса. Хотя при этом рассматривается возможность выделения в состоянии системы комплектов эпизодов на основе <<смешивания>> упорядоченных по времени выдержки подкортежей требований к различным значениям структурных атрибутов эпизодов, необходимых для срабатывания элементов интерфейса, однако, сам выбор конкретных эпизодов и сама возможность такого выбора с полученным на первом этапе минимально возможным временем срабатывания могут оказаться иными даже при условии сохранения дисциплины выбора, индивидуально заданной для каждого элемента интерфейса. Полагаем, что фактически смогут реализоваться только максимальные из этих готовых к совместному срабатыванию комплектов элементов интерфейса. Реализуемый таким образом принцип максимально возможного параллелизма в поведении системы позволяет исключить из рассмотрения события с одним и тем же временем. Если таких максимальных суммарных комплектов более одного, то поведение системы становится недетерминированным, и, как было сказано ранее, оно формализуется в виде дерева процессов: происходит его ветвление, вершина текущего состояния системы будет иметь несколько потомков, по одному для каждого указанного выше случая. Каждый из вариантов определяет и свой результат соответствующего события, т.е. изменения состояния системы и действующего интерфейса. 

\section{Результат события}
Перейдем к краткому рассмотрению результата события – к изменениям состояния системы, а для динамических систем и действующего интерфейса. 

Пусть процесс не обрывается и $ t_{i+1} $  – время очередного события. 
Во-первых, из состояния $ P_i $ уже исключены эпизоды, время максимально возможного окончания которых меньше $ t_{i+1} $. 
Участвующие в событии эпизоды также исключаются из состояния системы. 
Каждый из участвовавших в событии элемент интерфейса порождает, с учетом его кратности вхождения в реализованный в событии комплект элементов интерфейса, комплект новых эпизодов, добавляемых к текущему состоянию системы (после указанных выше удалений эпизодов). 
Для каждого нового эпизода задаются полностью (и контейнеры, и сорта) структурные атрибуты, <<задержка>> (число из $ Q $ ), результат сложения которого с  $ t_{i+1} $  дает начало этого эпизода, и еще одно число из $ \breve{Q} $ . 
Если оно равно нулю, то длительность нового эпизода не ограничена сверху (параметр <<конец эпизода>> получает значение $ \omega $ ), в противном случае он задает положительную длительность эпизода, а значение параметра <<конец эпизода>> получается сложением длительности с началом эпизода. 
Все это, в сочетании с требованием задания положительных значений для выдержек, гарантирует отсутствие готовности к срабатыванию в момент времени $ t_{i+1} $ новых, появившихся в результате рассматриваемого события, эпизодов в состоянии системы. 
В общем случае все характеристики новых эпизодов определяются как значения заданных для каждого элемента интерфейса функций из функционального базиса системы, согласованных по сортам аргументов с сортами входных эпизодов элемента интерфейса. 
Значениями аргументов этих функций выступают значения информационных атрибутов выделенных в состоянии системы $ P_i $ участвующих в событии эпизодов для рассматриваемого элемента интерфейса. 
Именно с целью установления соответствия аргументов и их значений перечисление требований к эпизодам уже в элементе интерфейса осуществляется в некотором порядке, т.е. в форме кортежа. 
В частном случае, структурные атрибуты (контейнеры, сорта) и максимально возможные времена окончаний новых эпизодов задаются непосредственно в элементе интерфейса, если всем сортам эпизодов сопоставлены одноэлементные множества возможных значений информационных атрибутов. 

Описанные события не приводят к изменению интерфейса системы, множество используемых контейнеров ограничено множеством контейнеров, фигурирующих в начальном состоянии системы и ее интерфейсе. 
Такие системы будем называть статическими. 
Для описания поведения динамических систем, интерфейс которых может изменяться во времени, а множества используемых контейнеров неограниченно расширяться, нужны совсем иные пути формализации для реализации этих возможностей. 
Не претендуя на общность, мы предлагаем дополнительно ввести в качестве элементов интерфейсов динамических систем так называемые нетерминальные элементы, в то время как рассмотренные ранее будем называть терминальными элементами интерфейсов. 

Если для терминальных элементов интерфейса эффект выражается в создании новых эпизодов и сохранении действующего интерфейса, то в результате срабатывания нетерминального элемента интерфейса происходят структурные изменения: могут появиться новые элементы интерфейса, могут появиться не только новые эпизоды в состоянии системы, но и новые контейнеры. 
Вопрос же об удалении из рассмотрения некоторых контейнеров и элементов интерфейса должен решаться подобно тому, как решается проблема <<сборки мусора>> для памяти типа <<куча>>.

Динамическая система представлена в виде конечного множества именованных подсистем, каждая из которых задана тройкой~--- начальным состоянием, начальным интерфейсом и упорядоченным конечным подмножеством контейнеров подсистемы, элементы которого будем называть контактами подсистемы. 
Контейнеры подсистемы, не вошедшие в число контактов, при срабатывании нетерминального элемента интерфейса и подключении к системе этой подсистемы будут представлять новые контейнеры, отличные от всех ранее используемых в системе. 
Механизм порождения новых контейнеров аналогичен процессу выделения новых ячеек в памяти типа <<куча>> и, в какой-то мере, операции замены связанной переменной в формальных системах с операторами, связывающих вхождения операторной переменной в терм, к которому применяется оператор (например в $ \lambda $-исчислении). 
Вершина дерева процессов (начальное состояние системы) представлена начальным состоянием выделенной базовой подсистемы (аксиомы). 
Предполагается возможность в дальнейшем подключений к системе любых подсистем, в том числе и подсистемы-аксиомы. 
Множество имен всех подсистем образуют особый сорт $ D $ значений информационных атрибутов эпизодов.

В текущем интерфейсе всякий нетерминальный элемент представлен следующей информацией: 
\begin{enumerate}[label=1)]
	\item контейнером, наличие в котором эпизода сорта $ D $   может привести к срабатыванию этого нетерминального элемента, 
	\item  выдержкой, играющей ту же роль, что и выдержки эпизодов в терминальных элементах интерфейса, 
	\item  кортежем контейнеров текущего состояния системы, элементы которого будем называть контактами рассматриваемого нетерминального элемента текущего интерфейса (заметим, что один и тот же контейнер может входить неоднократно в указанный кортеж), 
	\item  функцией, аргументом которой является имя подсистемы, а значением – задержка подключения соответствующей подсистемы из множества $ Q $ .
\end{enumerate}  

Условием возможного срабатывания нетерминального элемента интерфейса в момент времени $ t_{i+1} = t_i + \delta $ , где $ \delta $~--- указанная для этого элемента <<выдержка>>, является просто наличие в состоянии системы соответствующего эпизода сорта $ D $ в указанном контейнере. 
В результате срабатывания нетерминального элемента интерфейса происходит подключение к системе некоторой подсистемы, имя которой определяется информационным атрибутом эпизодов, что приводит к изменениям основных компонентов состояния системы, помимо тех, которые вносятся в результате возможного одновременного срабатывания терминальных элементов интерфейса. 
Одна из основных проблем определения процедуры подключения подсистемы состоит в том, чтобы она исключала возможность в момент времени подключения срабатывания новых элементов обновленного интерфейса. 
Кроме того, при одновременном срабатывании нескольких нетерминальных элементов результат подключения нескольких подсистем не должен зависеть от порядка их реализации.

\subsection{Подключение подсистемы}
Опишем вначале способ подключения к системе некоторой одной подсистемы.

Как было сказано ранее, для выполнения подключений каждая подсистема, помимо ее начального состояния и начального интерфейса, содержит контактную информацию для ее подключений, и, в свою очередь, всякий нетерминальный элемент действующего интерфейса системы также содержит свою контактную информацию. 
Контактная информация позволяет установить связь между контейнерами основной системы и контейнерами подключаемой подсистемы путем замены контейнеров в подключаемой системе на соответствующие (согласно порядкам перечисления контактов) контейнеры основной системы; 
именно для того, чтобы при подключении не возникла возможность <<ложных>>, не реализованных к моменту подключения срабатываний элементов действующего интерфейса системы, соответствующая контактная информация в основной системе задается в форме кортежа используемых в ней контейнеров, а в подключаемой подсистеме --- в форме упорядоченного подмножества используемых в ней контейнеров, причем для остальных контейнеров подключаемой подсистемы производится их замена на новые контейнеры, не используемые ранее в основной системе. 
При подключении подсистемы начала всех эпизодов во всех ее контейнерах формируются путем сложений их начал и, соответственно, концов, заданных в начальном состоянии подключаемой подсистемы, времени события, включающего срабатывание рассматриваемого нетерминального элемента основной системы, и задержки подключения соответствующей подсистемы при срабатывании конкретного нетерминального элемента интерфейса системы. 
Так как новый интерфейс системы в результате срабатывания нетерминального элемента пополняется элементами интерфейса подключаемой подсистемы, то указанное изменение временных параметров подключаемых к состоянию системы эпизодов подсистемы не может привести к готовности в тот же момент времени к срабатыванию как <<старых>>, так и <<новых>> элементов интерфейса. 
Очевидно, что, так как интерфейс подключаемых подсистем может тоже содержать нетерминальные элементы, то в системе может неограниченно увеличиваться количество используемых контейнеров и элементов интерфейса, а описанная далее <<сборка мусора>> в системе может приводить к удалению <<лишних>> эпизодов, контейнеров и элементов интерфейса.

В общем случае, событие, приводящее к изменению состояния системы в момент времени $ t_{i+1} $ , может включать одновременное срабатывание не только нескольких терминальных элементов интерфейса, но и нескольких его нетерминальных элементов. 
Подключение одной подсистемы предполагает следующую последовательность шагов:
\begin{itemize}
	\item создание рабочих копий начального состояния и начального интерфейса выбранной для подключения подсистемы (далее под состоянием и интерфейсом подключаемой подсистемы будем понимать эти копии); 
	изменение в состоянии и интерфейсе подключаемой подсистемы временных параметров эпизодов так, как было описано выше;
	\item замена различных контейнеров в состоянии и интерфейсе подключаемой подсистемы на различные новые контейнеры, не используемые ни в текущем состоянии и интерфейсе системы, ни в начальном состоянии и интерфейсе подключаемой подсистемы;
	\item замена всех контейнеров в состоянии и интерфейсе подключаемой подсистемы, входящих в число ее контактов, на соответствующие им по порядку контейнеры в кортеже контактов рассматриваемого нетерминального элемента интерфейса (если в последнем <<хватает>> контактов);
	\item к состоянию системы присоединяется состояние подключаемой подсистемы (путем сложения представляющих их комплектов), а к интерфейсу систем добавляется интерфейс подключаемой подсистемы (путем конкатенации представляющих их кортежей).
\end{itemize}

\section{Вывод}
В своей работе мы предполагаем и наличие принципиальных отличий даже базовых понятий: 
\begin{itemize}
	\item отказ от переходов с пустым комплектом входных позиций: в этом отношении мы обобщаем принцип <<отсутствие необходимых ресурсов (фишек) не может служить основанием каких-либо событий (переходов)>>. Это требование, не являясь принципиальным при рассмотрении поведения системы в абстрактном времени, для реального времени является принципиальным;
	\item введение линейных порядков на множествах входных и выходных позиций переходов уже используются во многих обобщениях формализма сетей Петри, и, как следствие, замена понятий <<комплект входных (выходных) позиций перехода>> на понятия <<кортеж входных (выходных) эпизодов для элемента интерфейса>>. В  нашем формализме это необходимо для описания роли информационных атрибутов эпизодов;
	\item срабатывание переходов (элементов интерфейса) управляется не только наличием фишек во входных позициях (контейнерах), а и их окраской (сортами эпизодов);
	\item изменение состояний происходят в реальном, а не в абстрактом (модельном) времени, что позволяет строить более адекватные модели поведения реальных систем;
	\item в нашем подходе всегда реализуется максимально возможный параллелизм срабатывания элементов интерфейса (в базовой модели сетей Петри не допускается одновременное срабатывание нескольких переходов, а моделирование этого делает сеть неоправданно громоздкой). Этот подход позволяет в системе реального времени избавиться от возможности нескольких событий с одинаковым временем;
	\item наконец, главное: сами рассматриваемые нами системы являются динамическими, могут структурно меняться во времени и иметь при функционировании априори не ограниченную сложность.
\end{itemize}

Изменением терминологии мы хотим подчеркнуть следующую основные цели нашей публикации: описание проходящих в реальном времени параллельных вычислительных процессов в динамических системах с априори не ограниченной сложностью\cite{Falk}.