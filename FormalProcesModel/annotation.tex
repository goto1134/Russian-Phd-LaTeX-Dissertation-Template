\section*{Аннотация}
Работа посвящена рассмотрению формализма процессов реального времени, анализу требований к программному обеспечению, позволяющему моделировать процесс реального времени, а также разработке такой программы для статических систем. Для достижения данной цели проведен анализ модели, планирование разработки, разработана и протестирована программа, осуществляющая моделирование, а также создан предметно-ориентированный язык описания вычислительных систем.

Полный объём диссертации составляет \formbytotal{TotPages}{страниц}{у}{ы}{} 
с~\formbytotal{totalcount@figure}{рисунк}{ом}{ами}{ами}
и~\formbytotal{totalcount@table}{таблиц}{ей}{ами}{ами}. Список литературы содержит  
\formbytotal{citenum}{наименован}{ие}{ия}{ий}.
\section*{Annotation}
The paper is devoted to consideration of real-time process formalism,
analysis of the requirements for the software that allows you to simulate real-time processes, as well as the development of a program that simulates process of static systems. To achieve this, the analysis of the model was made, development planning performed, and simulation program was developed and tested.
In addition, a domain-specific language for computational system specification was developed.


This tesis has volume of \formbytotal{TotPages}{pages}{}{}{} 
with~\formbytotal{totalcount@figure}{figures}{}{}{}
and~\formbytotal{totalcount@table}{tables}{}{}{}. References contain
\formbytotal{citenum}{items}{}{}{}.